\documentclass[10pt]{beamer}

\usepackage[ngerman]{babel}
\usepackage{amsmath, amssymb}

\usetheme{Boadilla}

\author[\url{https://fdf-uni.github.io/ft}]{}
\title{Tutorium 8}
\subtitle{\texorpdfstring{Funktionentheorie\vspace*{-1.5cm}}{Funktionentheorie}}
\date{23. bis 25. Juni 2025}

\newcommand{\iu}{\mathrm{i}}
\renewcommand{\Re}{\operatorname{Re}}
\renewcommand{\Im}{\operatorname{Im}}

\begin{document}
\begin{frame}
	\titlepage
\end{frame}
\begin{frame}
	\frametitle{Das Argumentprinzip}
	\begin{theorem}
		Sei $f$ holomorph in einer Umbegung einer Kreisscheibe $\overline{D}$, bis auf endlich viele Pole.
		Falls $f$ auf $C = \partial D$ weder Pole noch Nullstellen hat, so gilt
		\[
			\frac{1}{2 \pi \iu} \int_{C} \frac{f'(z)}{f(z)} \,\mathrm{d}z = \# N - \# P,
		\]
		wobei $\# N$ bzw. $\# P$ die Anzahl der Null- bzw. Polstellen von $f$ innerhalb von $C$ bezeichnet, jeweils mit Vielfachheit gezählt.
	\end{theorem}
	\pause
	Merkhilfe: $\frac{f'}{f}$ heißt \emph{logarithmische Ableitung}.\\
	\pause
	Warum Argumentprinzip? \pause $\leadsto$ s. Animation.
\end{frame}
\begin{frame}
	\frametitle{Der Satz von Rouché}
	\pause
	\begin{theorem}
		Seien $f$ und $g$ holomorph in Umgebung einer Kreisscheibe $\overline{D}$.
		Gilt zusätzlich
		\[
			\lvert f(z) \rvert > \lvert g(z) \rvert \qquad \text{für alle } z \in \partial D,
		\]
		so besitzen $f$ und $f + g$ (mit Vielfachheit) gleich viele Nullstellen innerhalb von $D$.
	\end{theorem}
\end{frame}
\begin{frame}
	\frametitle{Offenheits- und Maximumprinzip}
	\pause
	\begin{theorem}[Satz von der offenen Abbildung]
		Seien $\Omega \subset \mathbb{C}$ offen und zusammenhängend und $f \colon \Omega \to \mathbb{C}$ eine holomorphe, nicht-konstante Funktion.
		Dann ist $f$ eine offene Abbildung, d.h. $f(U)$ ist offen für alle $U \subset \Omega$ offen.
	\end{theorem}
	\pause
	\begin{theorem}[Maximumprinzip]
		Seien $\Omega \subset \mathbb{C}$ offen und zusammenhängend und $f \colon \Omega \to \mathbb{C}$ eine holomorphe, nicht-konstante Funktion.
		Dann nimmt $\lvert f \rvert$ auf $\Omega$ nicht sein Maximum an.

		Insbesondere gilt, falls $\Omega$ beschränkt ist und $f$ stetig auf $\overline{\Omega}$ fortgesetzt werden kann:
		\[
			\sup_{\Omega} \lvert f \rvert = \sup_{\partial \Omega} \lvert f \rvert.
		\]
	\end{theorem}
\end{frame}
\begin{frame}
	\frametitle{Der (oder vielleicht doch \emph{die}?) Logarithmus (Logarithmen?)}
	\begin{theorem}
		Sei $\Omega \subset \mathbb{C}$ offen und einfach zusammenhängend und sei $f \colon \Omega \to \mathbb{C}$ holomorph mit $f(z) \neq 0$ für alle $z \in \Omega$.
		Dann existiert eine holomorphe Abbildung $g \colon \Omega \to \mathbb{C}$ mit
		\[
			f(z) = \mathrm{e}^{g(z)} \qquad \forall z \in \Omega.
		\]
		\pause
		Die Funktion $g$ ist eindeutig bis auf Addition von $2 \pi \iu k$ mit $k \in \mathbb{Z}$.
	\end{theorem}
	\pause
	\begin{corollary}
		Sei $\Omega \subset \mathbb{C}$ offen und einfach zusammenhängend mit $0 \notin \Omega$ und sodass ein $R \in \Omega \cap (0, \infty)$ existiert.
		Dann existiert eine holomorphe Funktion $g$ in $\Omega$ mit $z = \mathrm{e}^{g(z)}$ für alle $z \in \Omega$ und sodass $g(r) = \log(r)$ für $r \in \Omega \cap (0, \infty)$ nahe $R$.
	\end{corollary}
	\pause
	Der sogenannte \emph{Hauptzweig des Logarithmus} ist in der geschlitzten Ebene $\mathbb{C} \setminus (-\infty, 0]$ definiert als
	\[
		\operatorname{Log} z = \ln r + \iu \theta \text{ für } z = r \mathrm{e}^{\iu \theta}, \ \theta \in (- \pi, \pi).
	\]
\end{frame}
\end{document}
