\documentclass[10pt]{beamer}

\usepackage[ngerman]{babel}
\usepackage{amsmath, amssymb}

\usetheme{Boadilla}

\author[\url{https://fdf-uni.github.io/ft}]{}
\title{Tutorium 3}
\subtitle{\texorpdfstring{Funktionentheorie\vspace*{-1.5cm}}{Funktionentheorie}}
\date{12. \& 13. Mai 2025}

\newcommand{\iu}{\mathrm{i}}
\renewcommand{\Re}{\operatorname{Re}}
\renewcommand{\Im}{\operatorname{Im}}

\begin{document}
\begin{frame}
	\titlepage
\end{frame}
\begin{frame}
	\frametitle{Ergänzung zum letzten Mal}
	\pause
	Das Wurzelkriterium ist \emph{strikt} stärker als das Quotientenkriterium: Für
	\[
		a_n := \begin{cases} 2^{-n} & \text{falls } n \text{ gerade}, \\ 2^{-n + 1} & \text{falls } n \text{ ungerade}, \end{cases}
	\]
	\pause
	haben wir $\frac{1}{2} \le \sqrt[n]{\lvert a_n \rvert} \le \frac{1}{2} \sqrt[n]{2} \overset{n \to \infty}{\longrightarrow} \frac{1}{2}$, also gilt gemäß \glqq Sandwich-Lemma\grqq{} $\lim_{n \to \infty} \sqrt[n]{\lvert a_n \rvert} = \frac{1}{2}$.
	\pause
	Allerdings ist
	\[
		\frac{\lvert a_{2n + 1} \rvert}{\lvert a_{2n} \rvert} = \frac{2^{-(2n + 1) + 1}}{2^{- 2 n}} = 1 \qquad \text{und} \qquad \frac{\lvert a_{2n + 2} \rvert}{\lvert a_{2n + 1} \rvert} = \frac{2^{-(2n + 2)}}{2^{-(2n + 1) + 1}} = \frac{1}{4},
	\]
	d.h. $\frac{\lvert a_{n + 1} \rvert}{\lvert a_n \rvert}$ konvergiert \emph{nicht}.
	Das Quotientenkriterium ist hier also im Gegensatz zum Wurzelkriterium nicht anwendbar.
\end{frame}
\begin{frame}
	\frametitle{Kurven}
	\pause
	Für Definition einer Kurve und Eigenschaften wie \emph{geschlossen}, \emph{einfach} oder \emph{äquivalent}, s. Vorlesung (im Wesentlichen ist eine Kurve eine stetige Abbildung $z \colon [a, b] \to \Omega \subset \mathbb{C}$ mit $z \in C_p^1([a, b])$).
	\pause
	\begin{definition}[Kurve mit umgekehrter Orientierung]
		$z^{-} \colon [a, b] \to \mathbb{C}, \ t \mapsto z(b + a - t)$.
	\end{definition}
\end{frame}
\begin{frame}
	\frametitle{Integration entlang von Kurven}
	\begin{definition}[Kurvenintegral]
		Gegeben einer Kurve $\gamma$ mit Parametrisierung $z \colon [a, b] \to \mathbb{C}$ und \glqq{}Knickstellen\grqq{} $a = a_0 < \ldots < a_K = b$ definieren wir das \emph{Integral von $f$ entlang $\gamma$} mittels
		\[
			\int_{\gamma} f(z) \,\mathrm{d}z := \sum_{k=1}^{K} \int_{a_{k-1}}^{a_k} f(z(t)) z'(t) \,\mathrm{d}t.
		\]
	\end{definition}
	\pause
	Merkregel für/Intuition zum Faktor $z'(t)$ (rein heuristisch):
	\pause
	\begin{itemize}
		\item \glqq{}Wie bei der Substitutionsregel\grqq{}
		      \pause
		\item Bezieht \glqq{}Geschwindigkeit\grqq{} der Parametrisierung mit ein
		      \pause
		\item Unabhängigkeit von Parametrisierung (vorausgesetzt, parametrisierte Kurven sind äquivalent im Sinne der Definition aus der Vorlesung)
	\end{itemize}
\end{frame}
\begin{frame}
	\frametitle{(Holomorphe) Stammfunktionen}
	\begin{definition}
		Sei $\Omega \subset \mathbb{C}$ offen und $f \colon \Omega \to \mathbb{C}$.
		Dann heißt $F \colon \Omega \to \mathbb{C}$ \emph{Stammfunktion (von $f$)}, falls $F$ holomorph in $\Omega$ ist mit $F' = f$.
	\end{definition}
	\pause
	\begin{theorem}
		Mit $\Omega$, $f$ und $F$ wie oben, gilt für jede Kurve $\gamma$ in $\Omega$ mit Anfangspunkt $w_0$ und Endpunkt $w_1$, dass
		\[
			\int_{\gamma} f(z) \,\mathrm{d}z = F(w_1) - F(w_0).
		\]
		Insbesondere gilt, für jede geschlossene Kurve $\gamma$ (also mit $w_0 = w_1$):
		\[
			\int_{\gamma} f(z) \,\mathrm{d}z = 0.
		\]
	\end{theorem}
\end{frame}
\end{document}
