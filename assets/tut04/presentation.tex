\documentclass[10pt]{beamer}

\usepackage[ngerman]{babel}
\usepackage{amsmath, amssymb}
\usepackage{pgfplots}
\usepackage{qrcode}

\pgfplotsset{compat=1.18}
\usetikzlibrary{decorations.markings, hobby, shapes}
\tikzset{tikzPathWithMidArrow/.style={draw=#1,postaction={decorate, decoration={show path construction, lineto code={\path [postaction={decorate,decoration={markings, mark=at position .5 with {\arrow[#1]{stealth}}}}] (\tikzinputsegmentfirst) -- (\tikzinputsegmentlast);},curveto code={\path [postaction={decorate,decoration={markings, mark=at position .5 with {\arrow[#1]{stealth}}}}] (\tikzinputsegmentfirst) .. controls (\tikzinputsegmentsupporta) and (\tikzinputsegmentsupportb) .. (\tikzinputsegmentlast);}}}}}

\usetheme{Boadilla}

\author[\url{https://fdf-uni.github.io/ft}]{}
\title{Tutorium 4}
\subtitle{\texorpdfstring{Funktionentheorie\vspace*{-1.5cm}}{Funktionentheorie}}
\date{26. \& 27. Mai 2025}

\newcommand{\iu}{\mathrm{i}}
\renewcommand{\Re}{\operatorname{Re}}
\renewcommand{\Im}{\operatorname{Im}}

\begin{document}
\begin{frame}
	\titlepage
\end{frame}
\begin{frame}
	\frametitle{Organisatorisches}
	\begin{center}
		\qrcode[height=6cm, level=H]{https://fdf-uni.github.io/ft}

		\vspace*{0.3cm}
		\noindent\Large\href{https://fdf-uni.github.io/ft}{https://fdf-uni.github.io/ft}
	\end{center}
\end{frame}
\begin{frame}
	\frametitle{Homotope Kurven}
	\begin{definition}
		Sei $\Omega \subset \mathbb{C}$ offen und seien $\gamma_0,\gamma_1$ zwei Kurven in $\Omega$, die im gleichen Punkt $\alpha$ beginnen und im gleichen Punkt $\beta$ enden. Die Kurven $\gamma_0$ und $\gamma_1$ heißen \emph{homotop} in $\Omega$,wenn es für alle Parametrisierungen $z_0,z_1:[a,b]\to\Omega$ von $\gamma_0$ und $\gamma_1$ eine stetige Abbildung $\Gamma:[a,b]\times[0,1]\to\Omega,(t,s)\to\Gamma_s(t)$ gibt mit
		\begin{align*}
			\Gamma_s(a) & = \alpha, & \Gamma_s(b) & =\beta  & \forall s\in[0,1] \\
			\Gamma_0(t) & = z_0(t), & \Gamma_1(t) & =z_1(t) & \forall t\in[a,b]
		\end{align*}
	\end{definition}
	\pause
	\begin{center}
		\begin{tikzpicture}
			\draw[dotted, closed hobby] plot coordinates {(0,0) (.5,-.5) (1,-.5) (2,-.5) (2.5,.5) (3,2) (2,2.5) (1,2.7) (0,1.5)};
			\node[label=left:$\alpha$,circle, inner sep=0pt, minimum size=3pt, fill] (alpha) at (0.2,0.2) {};
			\node[label=right:$\beta$,circle, inner sep=0pt, minimum size=3pt, fill] (beta) at (2.85,1.85) {};

			\path[tikzPathWithMidArrow, orange]
			(alpha).. controls (.8,3) and (2.4,2.6)..(beta)
			(alpha).. controls (.8,-0.8) and (2.5,0) ..(beta);
			\foreach \s in {.1,.3,.5,.7,.9}{
					\pgfmathsetmacro\firstY{(1-\s)*3+\s*(-1)}
					\pgfmathsetmacro\secondY{(1-\s)*2.5+\s*0}
					\path[tikzPathWithMidArrow,dashed]
					(alpha).. controls (.8,\firstY) and (2.5,\secondY) .. (beta);
				}

			\node[above] (gam_0) at (.8,1.9) {$\gamma_0$};
			\node[below] (gam_1) at (2.1,.1) {$\gamma_1$};
			\node[label=above:$\Gamma_s(t)$,circle, inner sep=0pt, minimum size=3pt, fill] (beta) at (1.6,1.58) {};
			\node[above] (Omega) at (.4,2.5) {$\Omega$};
		\end{tikzpicture}
	\end{center}
\end{frame}
\begin{frame}
	\frametitle{Homotopieinvarianz des Kurvenintegrals}
	\begin{theorem}
		Sei $\Omega \subset \mathbb{C}$ offen, $f:\Omega\to \mathbb{C}$ holomorph und seien $\gamma_0$, $\gamma_1$ in $\Omega$ homotope Kurven.
		Dann ist
		\[
			\int_{\gamma_0} f(z) \,\mathrm{d}z = \int_{\gamma_1} f(z) \,\mathrm{d}z.
		\]
	\end{theorem}
\end{frame}
\begin{frame}
	\frametitle{Zusammenhang}
	\pause
	\begin{definition}
		Eine offene Menge $\Omega \subset \mathbb{C}$ heißt \emph{zusammenhängend}, falls es nicht zwei nicht-leere, disjunkte, offene Mengen $U_1, U_2 \subseteq \mathbb{C}$ mit $\Omega = U_1 \cup U_2$ gibt.\\
		Sie heißt \emph{wegzusammenhängend}, falls für alle $z_1, z_2 \in \Omega$ eine stetige Kurve $\gamma \colon [0, 1] \to \mathbb{C}$ mit $\gamma(0) = z_1$ und $\gamma(1) = z_2$ existiert.
	\end{definition}
	\pause
	\emph{Bemerkung:} Für $\Omega \subset \mathbb{C}$ offen sind zusammenhängend und wegzusammenhängend äquivalent. (Allgemein gilt nur \glqq$\Leftarrow$\grqq{} in topologischen Räumen, \glqq$\Rightarrow$\grqq{} benötigt Zusatzvoraussetzungen.)
	\pause
	\begin{definition}
		Unter einem \emph{Gebiet} verstehen wir eine nicht-leere, offene, zusammenhängende Menge $\Omega \subset \mathbb{C}$.
	\end{definition}
\end{frame}
\begin{frame}
	\frametitle{Einfach zusammenhängende Mengen}
	\pause
	\begin{definition}
		Eine zusammenhängende, offene Menge $\Omega \subset \mathbb{C}$ heißt \emph{einfach zusammenhängend}, falls je zwei Kurven mit denselben Endpunkten homotop in $\Omega$ sind (äquivalent: jede geschlossene Kurve ist homotop zu einer konstanten Kurve, man sagt auch, sie ist \emph{nullhomotop}).
	\end{definition}
	\pause
	Intuition: Man kann in der Menge jedes \glqq Lasso zusammenziehen\grqq:

	\begin{center}
		\begin{tikzpicture}
			\filldraw[dotted, closed hobby, fill=black!5] plot coordinates {(0,0) (.5,-.5) (1,-.5) (2,-.5) (2.5,.5) (3,2) (2,2.5) (1,2.7) (0,1.5)};
			\node[above] (Omega) at (.4,2.5) {$\Omega_1$};
			\pause
			\node[] at (1.5,-1.5) {einfach zusammenhängend};
			\pause

			\path[tikzPathWithMidArrow]
			(0.2,0.2).. controls (.8,3) and (2.4,2.6).. (2.85,1.85) .. controls (2.5,0) and (.8,-0.8) .. (0.2,0.2);
			\pause
			\path[tikzPathWithMidArrow, dashed]
			(0.5, 0.5).. controls (.8,2.2) and (2.3,2).. (2.35,1.35) .. controls (2.5,0.7) and (.8,-0.1) .. (0.5, 0.5);
			\pause
			\path[tikzPathWithMidArrow, dashed, rotate around={23:(1.4,1.1)}] (1.4,1.1) ellipse (0.2 and 0.12);
			\pause
			\fill[] (1.4, 1.1) circle (1pt);
			\pause

			\begin{scope}[xshift=5cm]
				\filldraw[dotted, closed hobby, fill=black!5] plot coordinates {(0,0) (.5,-.5) (1,-.5) (2,-.5) (2.5,.5) (3,2) (2,2.5) (1,2.7) (0,1.5)};
				\filldraw[dotted, fill=white] (1.4, 1.1) circle (0.5);
				\node[above] (Omega) at (.4,2.5) {$\Omega_2$};
				\pause
				\node[] at (1.5,-1.5) {nicht einfach zusammenhängend};
				\pause
				\path[tikzPathWithMidArrow]
				(0.2,0.2).. controls (.8,3) and (2.4,2.6).. (2.85,1.85) .. controls (2.5,0) and (.8,-0.8) .. (0.2,0.2);
				\pause
				\path[tikzPathWithMidArrow, dashed]
				(0.5, 0.5).. controls (.8,2.2) and (2.3,2).. (2.35,1.35) .. controls (2.5,0.7) and (.8,-0.1) .. (0.5, 0.5);
				\pause
				\draw[dashed, red] (1.4, 1.1) circle (0.55);
			\end{scope}
		\end{tikzpicture}
	\end{center}
\end{frame}
\begin{frame}
	\frametitle{Cauchysche Integralformel}
	\pause
	\begin{theorem}
		Seien $\Omega \subset \mathbb{C}$ offen und $f \colon \Omega \to \mathbb{C}$ holomorph.
		Dann ist $f$ beliebig oft (komplex) differenzierbar in $\Omega$ und für jede Kreisscheibe $D$ mit $\overline{D} \subset \Omega$ gilt mit $C := \partial D$, $z \in D$ und $n \in \mathbb{N}_0$, dass
		\[
			f^{(n)}(z) = \frac{n!}{2 \pi \iu} \int_{C} \frac{f(\zeta)}{(\zeta - z)^{n + 1}} \,\mathrm{d}\zeta.
		\]
	\end{theorem}
	\pause
	Hieraus folgt beispielsweise auch folgende Mittelwertseigenschaft (für geeignete $z$ und $r$):
	\[
		f(z) = \frac{1}{2 \pi} \int_{0}^{2 \pi} f(z + r \mathrm{e}^{\iu \varphi}) \,\mathrm{d}\varphi.
	\]
\end{frame}
\begin{frame}
	\frametitle{Holomorphe Funktionen sind analytisch}
	\pause
	\begin{theorem}
		Sei $\Omega \subset \mathbb{C}$ und $f \colon \Omega \to \mathbb{C}$ holomorph.
		Dann ist $f$ analytisch in $\Omega$ und für jedes $z_0 \in \Omega$ gilt
		\[
			f(z) = \sum_{n = 0}^{\infty} \frac{f^{(n)}(z_0)}{n!}(z - z_0)^n, \qquad \lvert z - z_0 \rvert < \operatorname{dist}(z_0, \partial \Omega).
		\]
	\end{theorem}
	\pause
	\begin{theorem}[Identitätssatz]
		Sei $\Omega \subset \mathbb{C}$ offen und zusammenhängend und seien $f, g \colon \Omega \to \mathbb{C}$ analytisch in $\Omega$.
		Falls es eine konvergente Folge $(w_k) \subset \Omega$ von paarweise verschiedenen Punkten mit Grenzwert in $\Omega$ gibt, sodass $f(w_k) = g(w_k)$ für alle $k$, so gilt $f(z) = g(z)$ für alle $z \in \Omega$.
	\end{theorem}
\end{frame}
\end{document}
