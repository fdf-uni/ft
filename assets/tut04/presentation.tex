\documentclass[10pt]{beamer}

\usepackage[ngerman]{babel}
\usepackage{amsmath, amssymb}
\usepackage{tikz}

\usetikzlibrary{shapes}

\usetheme{Boadilla}

\author[\url{https://fdf-uni.github.io/ft}]{}
\title{Tutorium 4}
\subtitle{\texorpdfstring{Funktionentheorie\vspace*{-1.5cm}}{Funktionentheorie}}
\date{19. \& 20. Mai 2025}

\newcommand{\iu}{\mathrm{i}}
\renewcommand{\Re}{\operatorname{Re}}
\renewcommand{\Im}{\operatorname{Im}}

\begin{document}
\begin{frame}
	\titlepage
\end{frame}
\begin{frame}
	\frametitle{Etwas zu Dreiecken}
	\pause
	\begin{theorem}
		Seien $D \subset \mathbb{C}$ eine offene Kreisscheibe und $f \colon D \to \mathbb{C}$ stetig mit
		\[
			\int_T f(z) \,\mathrm{d}z = 0
		\]
		für jedes Dreieck $T \subset D$.
		Dann besitzt $f$ eine Stammfunktion in $D$.
	\end{theorem}
	\pause
	\begin{lemma}[Goursat's lemma]
		Sei $\Omega \subset \mathbb{C}$ offen und $f \colon \Omega \to \mathbb{C}$ holomorph.
		Dann gilt für jedes Dreieck $T$, das mitsamt seinem \glqq Inneren\grqq{} in $\Omega$ enthalten ist,
		\[
			\int_T f(z) \,\mathrm{d}z = 0.
		\]
	\end{lemma}
\end{frame}
\begin{frame}
	\frametitle{Konvexe Mengen (in $\mathbb{C}$)}
	\pause
	\begin{definition}
		Eine Menge $\Omega \subset \mathbb{C}$ heißt \emph{konvex}, falls für alle $w_1, w_2 \in \mathbb{C}$ die Verbindungsstrecke $[w_1, w_2] := \{t w_1 + (1 - t) w_2 : t \in [0, 1]\}$ von $w_1$ nach $w_2$ vollständig in $\Omega$ enthalten ist:
		\[
			t w_1 + (1 - t) w_2 \in \Omega \qquad \forall t \in [0, 1]
		\]
	\end{definition}
	\pause
	\begin{center}
		\begin{tikzpicture}
			\draw[] (0, 0) circle (1.5);
			\coordinate[] (w1) at (-0.7, -0.3);
			\coordinate[] (w2) at (0.3, 0.5);
			\node[below left] at (w1) {$w_1$};
			\node[right] at (w2) {$w_2$};
			\fill[] (w1) circle (2pt);
			\fill[] (w2) circle (2pt);
			\draw[] (w1) -- (w2);
		\end{tikzpicture}
	\end{center}
\end{frame}
\begin{frame}
	\frametitle{Sternförmige Mengen (in $\mathbb{C}$)}
	\pause
	\begin{definition}
		Eine Menge $\Omega \subset \mathbb{C}$ heißt bezüglich eines Punktes $z_{0} \in \Omega$ \emph{sternförmig}, falls für jedes $w \in \Omega$ die Verbindungsstrecke $[z_{0}, w] := \{t z_{0} + (1 - t) w : t \in [0, 1]\}$ von $z_{0}$ nach $w$ vollständig in $\Omega$ enthalten ist:
		\[
			t z_{0} + (1 - t) w \in \Omega \qquad \forall t \in [0, 1]
		\]
	\end{definition}
	\pause
	\begin{center}
		\begin{tikzpicture}
			\node[star,draw, star points=5, star point height=1cm, star point ratio=2.5, minimum size=3cm] {};
			\node[label={[shift={(.2,-.5)}]$z_{0}$},circle, inner sep=0pt, minimum size=3pt, fill] (z*) at (0,0) {};
			\node[label=above:$z$,circle, inner sep=0pt, minimum size=3pt, fill] (z) at (0,.7) {};
			\draw[dotted] (z*)--(z);
			\pause
			\fill[red] (-0.5, -0.8) circle (2pt);
			\fill[red] (0.5, -0.8) circle (2pt);
			\draw[red, thick] (-0.5, -0.8) -- (0.5,-0.8);
			\node[align=center, red] at (0,-1.3) {nicht\\ konvex};
			\pause
			\begin{scope}[shift={(6,0)}]
				\draw[] (0,0) circle (1.5);
				\pause
				\draw[] (-0.318198051533946,-0.318198051533946)--(1.06066017177982,1.06066017177982);
				\pause
				\draw[densely dashed, cyan] (-0.318198051533946,-0.318198051533946)--(-1.06066017177982,-1.06066017177982);
				\node[label={[shift={(.2,-.5)}, color=cyan]$z_{0}$},circle, inner sep=0pt, minimum size=3pt, fill, cyan] (z*) at (-0.636396103067893,-0.636396103067893) {};
				\node[label=above:$z$,circle, inner sep=0pt, minimum size=3pt, fill] (z) at (-.3,.8) {};
				\draw[dotted] (z*)--(z);

			\end{scope}
		\end{tikzpicture}
	\end{center}
\end{frame}
\end{document}
