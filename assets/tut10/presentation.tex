\documentclass[10pt]{beamer}

\usepackage[ngerman]{babel}
\usepackage{amsmath, amssymb}

\usetheme{Boadilla}

\author[\url{https://fdf-uni.github.io/ft}]{}
\title{Tutorium 10}
\subtitle{\texorpdfstring{Funktionentheorie\vspace*{-1.5cm}}{Funktionentheorie}}
\date{14. und 15. Juli 2025}

\newcommand{\iu}{\mathrm{i}}
\renewcommand{\Re}{\operatorname{Re}}
\renewcommand{\Im}{\operatorname{Im}}

\newtheorem{proposition}{Proposition}

\begin{document}
\begin{frame}
	\titlepage
\end{frame}
\begin{frame}
	\frametitle{Unendliche Produkte}
	\pause
	\begin{definition}
		Sei $(a_n) \subset \mathbb{C}$.
		Wir sagen, dass das Produkt
		\[
			\prod_{n=1}^{\infty} (1 + a_n)
		\]
		gegen $L$ konvergiert, falls $L := \lim_{N \to \infty} \prod_{n = 1}^{N} (1 + a_n)$ existiert.
	\end{definition}
	\pause
	\begin{lemma}
		Falls $\sum_{n = 1}^{\infty} \lvert a_n \rvert < \infty$ gilt, so konvergiert $\prod_{n = 1}^{\infty} (1 + a_n)$ und der Grenzwert des Produkts ist genau dann $0$, wenn einer der Faktoren $0$ ist.
	\end{lemma}
\end{frame}
\begin{frame}
	\frametitle{Unendliche Produkte}
	\begin{proposition}
		Sei $\Omega \subset \mathbb{C}$ offen und $F_n \colon \Omega \to \mathbb{C}$, $n \in \mathbb{N}$, eine Folge holomorpher Funktionen.
		Falls $(c_n)_{n \in \mathbb{N}} \subset (0, \infty)$ existiert, sodass
		\[
			\forall z \in \Omega : \lvert F_n(z) - 1 \rvert \le c_n \qquad \text{und} \qquad \sum_{n = 1}^{\infty} c_n < \infty,
		\]
		so gilt:
		\begin{enumerate}
			\pause
			\item $\prod_{n = 1}^{\infty} F_n(z)$ konvergiert \emph{gleichmäßig} in $\Omega$ gegen eine holomorphe Funktion $G \colon \Omega \to \mathbb{C}$.
			      \pause
			\item Für jedes $z \in \Omega$ gilt $G(z) = 0$ genau dann, wenn $F_n(z) = 0$ für ein $n \in \mathbb{N}$.
			      \pause
			\item Falls $G(z) \neq 0$ für ein $z \in \Omega$ ist, so gilt
			      \[
				      \frac{G'(z)}{G(z)} = \sum_{n = 1}^{\infty} \frac{F_n'(z)}{F_n(z)}.
			      \]
		\end{enumerate}
	\end{proposition}
\end{frame}
\begin{frame}
	\frametitle{Das Eulerprodukt für $\sin$}
	\begin{theorem}[Euler]
		Es gilt
		\[
			\frac{\sin \pi z}{\pi} = z \prod_{n=1}^{\infty} \left(1 - \frac{z^2}{n^2} \right).
		\]
	\end{theorem}
	\pause
	Frage: Können wir auch andere holomorphe Funktionen \glqq faktorisieren\grqq?
\end{frame}
\begin{frame}
	\frametitle{Der Weierstraß'sche Produktsatz}
	\pause
	\begin{definition}[Weierstraß'sche Elementarfaktoren]
		Für $k \in \mathbb{N}_0$ definieren wir die folgenden holomorphen Funktionen:
		\[
			E_0(z) = 1 - z, \qquad E_k(z) := (1 - z) \mathrm{e}^{z + \frac{z^2}{2} + \dots + \frac{z^k}{k}}.
		\]
	\end{definition}
	\pause
	Motivation: Hat $f$ Nullstellen $(a_n)_{n \in \mathbb{N}}$, so konvergiert \glqq naive\grqq{} Faktorisierung $\prod_{k = 1}^{\infty} (a_n - z)$ nicht notwendigerweise!
	\pause
	Gibt es bessere Faktoren, sagen wir mit Nullstelle $1$, also als $1 - z$?
	\pause

	Offensichtlich hat jedes $E_k$ eine einfache Nullstelle bei $1$.
	\pause
	Zudem gilt für $z \in \mathbb{C}$ mit $\lvert z \rvert < 1$, dass
	\[
		1 - z = \exp(\operatorname{Log}(1 - z)) = \exp\left(- \sum_{k = 1}^{\infty} \frac{z^k}{k}\right).
	\]
	\pause
	Daher kann $E_k(z) = (1 - z) \exp\left(\sum_{k = 1}^{n} \frac{z^k}{k}\right)$ für $\lvert z \rvert \le 1$ beliebig nahe bei $1$ gewählt werden, vgl. Lemma 4.7, wodurch stets Konvergenz erreicht werden kann.
\end{frame}
\begin{frame}
	\frametitle{Der Weierstraß'sche Produktsatz}
	\pause
	\begin{theorem}[Weierstraß]
		Sei $(a_n)_{n \in \mathbb{N}} \subset \mathbb{C}$ mit $\lvert a_n \rvert \to \infty$.
		Dann existiert eine ganze Funktion $f$ mit Nullstellen genau bei den $a_n$ (mit Vielfachheit).
		Jede andere solche ganze Funktion  ist von der Form $f \mathrm{e}^g$ mit $g$ ganz.
	\end{theorem}
	\pause
	Der Beweis aus dem Skript liefert mit $m := \#\{n : a_n = 0\}$ und $(b_n)_{n \in \mathbb{N}}$ als der Folge $(a_n)_{n \in \mathbb{N}}$ ohne den Wert $0$, dass
	\[
		f(z) = z^m \prod_{n = 1}^{\infty} E_n\left(\frac{z}{b_n}\right).
	\]
\end{frame}
\begin{frame}
	\frametitle{Produktsatz von Hadamard}
	\pause
	\begin{definition}
		Eine ganze Funktion $f$ hat Wachstumsordnung $\le \rho$, falls Konstanten $A, B > 0$ existieren, sodass
		\begin{equation}\label{eq:growth_order}
			\lvert f(z) \rvert \le A \mathrm{e}^{B \lvert z \rvert^{\rho}}
		\end{equation}
		für alle $z \in \mathbb{C}$ gilt.
		\pause
		\emph{Die Wachstumsordnung} $\rho_0$ von $f$ ist das Infimum über alle $\rho$, für welche (\ref{eq:growth_order}) erfüllt ist.
	\end{definition}
	\pause
	\begin{theorem}[Hadamard]
		Seien $f$, $m$ und $(b_n)_{n \in \mathbb{N}}$ wie auf der vorherigen Folie.
		$f$ habe zudem Wachstumsordnung $\rho_0$ und $k \in \mathbb{Z}$ sei so gewählt, dass $k \le \rho_0 < k + 1$ (d.h. $k = \lfloor \rho_0 \rfloor$).
		Dann gilt
		\[
			f(z) = {\only<5-6>{\color{orange}}\mathrm{e}^{P(z)}} z^m \prod_{n = 1}^{\infty} E_{\only<6>{\color{red}}k} \left( \frac{z}{b_n} \right),
		\]
		wobei $P$ ein Polynom von Grad $\le k$ ist.
	\end{theorem}
\end{frame}
\end{document}
